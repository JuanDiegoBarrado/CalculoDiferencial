\documentclass[10pt,a4paper,openright]{book}

\title{CÁLCULO DIFERENCIAL}
\author{Juan Diego Barrado Daganzo e Iker Muñox Martínez\\2º de Carrera} %\\ es salto de linea
\date{\today}
\pagestyle{plain}
\setlength{\parskip}{0.35cm} %edicion de espaciado
\setlength{\parindent}{0cm} %edicion de sangría
\clubpenalty=10000 %líneas viudas NO
\widowpenalty=10000 %líneas viudas NO

\usepackage[spanish]{babel} %Para que el idioma por defecto sea español
\usepackage{amsmath} %Paquetes para mates
\usepackage{amsfonts} %Paquetes para mates
\usepackage{amssymb} %Paquetes para mates
\usepackage{latexsym} %Paquetes para mates
\usepackage{multicol} %Paquetes columnas
\usepackage{cancel} %Paquete tachar cosas
\usepackage[top=2.5cm, bottom=2.5cm, left=3cm, right=3cm]{geometry}

\usepackage{titlesec} %Formato de capitulos y secciones
	\titleformat{\chapter}[display]{\normalfont\huge\bfseries\color{capitulos}}{\thechapter}{20pt}{\Huge}[\titlerule{}]
	\titleformat{\section}{\normalfont\Large\bfseries\color{secciones}}{\thesection}{1em}{}
	\titleformat{\subsection}{\normalfont\large\bfseries\color{subsecciones}}{\thesubsection}{1em}{}
	\titleformat{\subsubsection}{\normalfont\normalsize\bfseries\color{subsubsecciones}}{\thesubsubsection}{1em}{}

\usepackage[dvipsnames,usenames]{color} %activar e incluir colores
	\definecolor{capitulos}{RGB}{60,0,0}%gama de colores de los capitulos
	\definecolor{secciones}{RGB}{95,8,5}%gama de colores de las secciones
	\definecolor{subsecciones}{RGB}{140,36,31}%gama de colores de las subsections
	\definecolor{subsubsecciones}{RGB}{188,109,79}%gama de colores de las subsubsections
	\definecolor{teoremas}{RGB}{164,56,32}
	
\usepackage{graphicx} %Para incluir fotos
\graphicspath{{./fotos/}}

\usepackage{pgfplots}
\usepackage{tkz-fct}

\usepackage{ntheorem}[thmmarks]	% paquete de formateo de entornos matemáticos

\theoremstyle{break}
\theoremheaderfont{\normalfont\bfseries\color{teoremas}}
\theorembodyfont{\itshape}
\theoremseparator{\vspace{0.2cm}}
\theorempreskip{\topsep}
\theorempostskip{\topsep}
\theoremindent0cm
\theoremnumbering{arabic}
\theoremsymbol{}
\theoremprework{\vspace{0.2cm} \hrule}
\theorempostwork{\vspace{0.2cm}\hrule}
	\newtheorem*{defi}{Definición}

\theoremprework{\vspace{0.25cm}}
	\newtheorem*{theo}{Teorema}

\theoremprework{\vspace{0.25cm}}
	\newtheorem*{coro}{Corolario}

\theoremprework{\vspace{0.25cm}}
	\newtheorem*{lema}{Lema}

\theoremprework{\vspace{0.25cm}}
	\newtheorem*{prop}{Proposición}

\theoremheaderfont{\normalfont}
\theorembodyfont{\normalfont}
\theoremsymbol{\hfill\square}
	\newtheorem*{demo}{\underline{Demostración}:}

\begin{document}
\maketitle

\chapter*{TEORÍA DE LA MEDIDA}
Vamos a introducir un elemento fundamental en esta asignatura, y muy útil para las posteriores, que no es otra cosa que la capacidad de definir qué es una medida y cómo podemos medir las cosas según un criterio general.

\section*{CONCEPTO DE MÉTRICA Y ESPACIOS NORMADOS}
En esta sección, se definen los elementos básicos para el estudio de medidas, distancias y se estudian las características de las estructuras que se generan a partir de dichas definiciones, con sus consecuentes resultados para otras áreas como la geometría.

\begin{defi}[distancia]
Sea $E\neq \emptyset$ un conjunto, decimos que $d: E\times E \rightarrow \mathbb R$ es una \textbf{métrica o distancia} siempre que se satisfaga las siguientes propiedades\footnote{Nótese que no está permitido valores que tiendan a infinito (comprendidos en $\bar{\mathbb R}$)}
\begin{itemize}
\item Positiva: $d(x,y)>0: \forall x, y \in E$
\item No degenerada: $d(x,y) > 0 \Leftrightarrow x = y$
\item Simetría: $d(x,y) = d(y,x): \forall x,y \in E$
\item Desigualdad triangular: $d(x,y)\leq d(x,z)+ d(z,y): \forall x,y,z\in E$
\end{itemize}

Al par $(E,d)$ lo denotamos como \textbf{espacio métrico}.

\end{defi}

\underline{\textbf{Ejemplos}}:
\begin{itemize}
\item Un ejemplo sencillo de comprobar es considerar $\mathbb R$ con la métrica $d(x,y) = |x-y|$ tradicional.

\item Otro ejemplo es, dado $E\neq \emptyset$ definimos la métrica discreta como:
$$d(x,y) = \begin{cases} 0 & x= y \\ 1 & x\neq y \end{cases}$$
Luego con la definición dada de distancia y un conjunto cualquiera $E$ tenemos que estos forman una peculiar definición de espacio métrico.	

\item Por último, definimos como \textbf{Espacio Euclídeo Usual} al conjunto:
$$\mathbb R ^n = \mathbb R\times \cdots \times \mathbb R = \{(x_1, \cdots, x_n): x_i\in \mathbb R\}$$
que junto con la suma y el producto por escalares usual, lo cual lo dota de estructura de espacio vectorial.
\end{itemize}

\begin{defi}[Espacios normados]
Sea $E$ un espacio vectorial real, se dice que $||\cdot||: E \rightarrow \mathbb R$ es una \textbf{norma} si se cumplen las siguientes propiedades:
\begin{itemize}
\item Positiva: $||x||\geq 0: \forall x \in E$
\item No degenerada: $||x|| = 0 \Leftrightarrow x = 0$
\item Homogénea: $||\lambda x|| = |\lambda| \cdot ||x||: \forall \lambda \in \mathbb R \wedge \forall x \in E$
\item Desigualdad triangular: $||x+y||\leq ||x|| + ||y||$
\end{itemize}
Al par $(E,||\cdot||)$ se le denota como \textbf{espacio normado}.
\end{defi}

\underline{\textbf{Ejemplos}}:
\begin{itemize}
\item Considerando $\mathbb R^n$ denominamos como la \textbf{clásica norma euclídea} a:
$$||x|| = ||x||_2 = \sqrt{x_1^2+\cdots + x_n^2} : \forall x \in \mathbb R^n $$
La demostración es trivial salvo el último apartado, que no se puede demostrar hasta más adelante puesto que se necesita de la \textit{Desigualdad de Cauchy-Schwarz}.
\end{itemize}

\begin{prop}[Métrica asociada a una norma]
Si consideramos $\left(E,||\cdot||\right)$ un espacio normado, podemos definir \textbf{la métrica asociada a dicha norma} como:
$$d_{||\cdot||}(x,y)=||x-y||: \forall x, y \in E$$
Cuya definición, por ser en base a una norma ya dada, hace fácilmente verificable las condiciones de métrica.
\end{prop}

\begin{defi}
De este modo, se define la \textbf{métrica o distancia euclídea} en $\mathbb R^n$ como la métrica asociada a la norma euclídea:
$$d(x,y)=d_2(x,y)=\sqrt{(x_1-y_1)^2+\cdots + (x_n-y_n)^2}$$
\end{defi}

\underline{\textbf{Ejemplos}}:
\begin{itemize}
\item Definimos el conjunto de las funciones continuas $E=\{f: [0,1]\rightarrow \mathbb R: f\mbox{ continua}\} = \mathbb C[0,1]$ y la función:
$$d(f,g) = \sup\{|f(x)-g(x)|: \forall x \in [0,1]\}$$
Lo primero de todo, comprobamos que está bien definida: como son funciones continuas la diferencia es continua y, por ser el valor absoluto una función continua, la composición con él también es continua. Consecuentemente, al tratarse de una función continua y acotada en $[0,1]$ el máximo se alcanza, así que, de hecho, no solo el supremo es un número finito, sino que es máximo de la función en ese intervalo.

Para ver que se trata de una norma, comprobamos la última propiedad (las anteriores son triviales):
$$d(f,g) = |f(x) - g(x)| = |f(x) - h(x) + h(x)- g(x)| \leq |f(x) - h(x)| + |h(x) - g(x)| \leq d(f,h) + d(h,g)$$

\item Junto al ejemplo anterior, podemos definir:
$$||f||_\infty = \sup\{|f(x)| : x \in [0,1]\}$$
y entonces el par $(E, || \ ||_\infty)$ es un espacio normado. Para comprobar que esto es una norma, se puede tratar dicha norma como $||f||_\infty = d(f,0)$ para demostrar sus propiedades.

\item Sea $E = \{f : (0,1] \rightarrow \mathbb{R} \mbox{ continua }\}$ definimos: 
$$d(f,g) = \sup\{|f(x) - g(x)| : \forall x \in (0,1]\}$$
Sin embargo, no es una métrica porque no está bien definida; por ejemplo:
$$d\left(\frac{1}{x}, 0\right) = \infty$$
\end{itemize}

\underline{\textbf{Observación}}:

Sin embargo, no toda métrica tiene asociada una norma en un espacio vectorial. Para verlo, tomamos $\mathbb R^n$ con la métrica discreta definida: observamos que no existe una norma que haga que el par $(\mathbb R^n, d)$ sea un espacio normado puesto que en caso de existir, por ejemplo, no se verifica la 3 propiedad:
$$\mbox{Sea }x\in \mathbb R: x\neq 0, \ \exists ||\cdot||\Rightarrow\underbrace{d(\lambda x , 0)}_{=1} = ||\lambda x|| = |\lambda|\cdot ||x|| = |\lambda| \cdot d(x,0) = |\lambda|$$
Y como debe ocurrir para todo lambda, es absurdo.

\begin{defi}[Norma ``\textit{p}'']
Consideramos en $\mathbb R^n$ la siguiente norma:
$$||x||_1=|x_1|+\cdots+|x_n|$$
En general, se define la norma $p$ (con $1<p<\infty$) como:
$$||x||_p = \left(|x_1|^p + \cdots + |x_n|^p\right)^{1/p}$$
Es decir, que en el caso extremo tenemos que:
$$||x||_\infty = \lim_{p\rightarrow\infty} ||x||_p = \max \{|x_j| : j = 1,\ldots, n\}$$
\end{defi}

\begin{defi}[Producto escalar]
Sea $E$ un espacio vectorial, se dice que $\langle \cdot, \cdot\rangle: E\times E \rightarrow \mathbb R$ es un \textbf{producto escalar} si es una forma bilineal definida positiva, es decir, cumple las propiedades:
\begin{itemize}
\item Definida positiva: $\langle x,x \rangle \geq 0 : \forall x \in E$
\item No degenerada: $\langle x,x \rangle = 0 \Leftrightarrow x = 0$
\item Homogeneidad: $\langle \lambda x, y \rangle = \lambda \langle x,y\rangle : \forall x, y \in E$
\item Bilinealidad: $\langle x,y+z\rangle = \langle x,y\rangle + \langle x,z\rangle : \forall x, y ,z \in E$
\item Simétrica: $\langle x,y\rangle = \langle y,x\rangle: ,\forall x, y ,\in E$
\end{itemize}
Al par $(E, \langle \cdot, \cdot\rangle)$ se le denomina espacio vectorial con producto escalar o espacio \textbf{pre-Hilbert}.
\end{defi}

\newpage

\underline{Ejemplos}
\begin{enumerate}
\item En $\mathbb{R}^n$, definimos $$\langle x,y\rangle = \sum_{j=1}^{n} x_j \cdot y_j : \forall x,y \in \mathbb{R}^n$$

\item En $\mathcal{C}[0,1]$, definimos $$\langle f,g \rangle = \int_{0}^{1} f(x) \cdot g(x) dx$$
Y es muy sencillo demostrar que ambas definiciones suponen un producto escalar en el espacio en el que están definidas.
\end{enumerate}

\begin{theo}[Desigualdad de Cauchy-Schwarz]
Sea $(E, \langle\cdot, \cdot\rangle)$ un espacio pre-Hilbert y sean $x,y \in E$ dos vectores cualesquiera. Entonces ocurre que:
$$|\langle x,y\rangle| \leq \sqrt{\langle x,x \rangle \cdot \langle y,y\rangle}$$
\end{theo}

\begin{demo}
\begin{itemize}
\item Si $x = 0$ o $y=0$, la desigualdad es trivial.

\item Tomemos en primer lugar un $\lambda\in \mathbb R$ arbitrario y escojamos $x,y \in E \setminus \{0\}$:
$$0 \leq \ \langle \lambda x + y, \lambda x + y\rangle = \lambda^2 \langle x,x\rangle + 2\lambda \langle x,y\rangle + \langle y,y\rangle$$
Esta ecuación\footnote{Esto es mayor o igual que 0 por la propiedad de definida positiva} describe una parábola que, a lo sumo, es tangente al eje $X$ pero nunca lo llega a cruzar porque es siempre $\geq 0$. Consecuentemente, el discriminante de esta ecuación nunca será estrictamente positivo ya que esto implicaría tener dos raíces, es decir:
$$\Delta = 4 \langle x,y\rangle^2 - 4\langle x,x\rangle\cdot \langle y,y\rangle \leq 0 \Leftrightarrow \langle x,y\rangle^2 -\langle x,x\rangle\cdot \langle y,y\rangle \leq 0 \Leftrightarrow|\langle x,y\rangle| \leq \sqrt{\langle x,x\rangle\cdot\langle y,y\rangle}$$
\end{itemize}
\end{demo}

\underline{\textbf{Observación}}:

Cabe destacar que hay igualdad si y solo si los vectores son proporcionales, es decir:
$$|\langle x,y\rangle| =\sqrt{\langle x,x\rangle \cdot \langle y,y\rangle} \Leftrightarrow x = \alpha y : \alpha \in \mathbb{R}$$
Si son proporcionales es trivial demostrar la igualdad, pero si tenemos la igualdad, entonces ello implica que la parábola de la que hablábamos antes corta en un único punto al eje de abscisas, luego:
$$\langle\lambda \cdot x + y, \lambda \cdot x + y\rangle =  0\Leftrightarrow  \lambda^2 \langle x,x\rangle + 2\lambda \langle x,y\rangle + \langle y,y\rangle = 0 \Leftrightarrow \lambda = \frac{-2 \cdot \langle x,y\rangle}{2 \cdot \langle x,x\rangle} \Leftrightarrow x = -\lambda y$$

\begin{prop}
Dado un espacio pre-Hilbert $(E, \langle \cdot, \cdot \rangle)$ y definimos $||x|| = \sqrt{\langle x,x \rangle}$, entonces $(E, || \cdot ||)$ es normado.
\end{prop}

\begin{demo}
La única propiedad no trivial es la desigualdad triangular. Sea 
$$|| x+y||^2 = \langle x+y,x+y\rangle = ||x||^2 + 2 \cdot \langle x,y\rangle + ||y||^2$$
Aplicando la desigualdad de Cauchy-Schwarz
$$\leq ||x||^2 + 2 ||x|| \cdot ||y|| + ||y||^2 = (||x|| + ||y||)^2$$

\underline{Ejemplos}
\begin{itemize}
\item En $\mathbb{R}^n$, si $x,y \in \mathbb{R}^n$, entonces:
$$\sum_{j=1}^{n} |x_j \cdot y_j | \leq \sqrt{\sum_{j=1}^{n} x_j^2} \cdot\sqrt{\sum_{j=1}^{n} y_j^2} $$

\item Sea $E = \ell^2 (\mathbb{N}) = \{x= \{x_j\} \in \mathbb{R} : ||x||^2_{\ell^2} = \sum_{n=1}^{\infty} x_j^2 < \infty\}$. Definimos 
$$\langle x,y \rangle = \sum_{j=1}^{\infty} x_j y_j$$

Comprobemos que se encuentra bien definida:
$$\sum_{j=1}^{\infty} |x_j y_j | \leq \sqrt{\sum_{j=1}^{n} x_j^2} \cdot\sqrt{\sum_{j=1}^{n} y_j^2} \leq \sqrt{\sum_{j=1}^{\infty} x_j^2} \cdot\sqrt{\sum_{j=1}^{\infty} y_j^2}  < \infty $$
$$\Rightarrow\sum_{j=1}^{\infty} |x_j|| y_j | \leq ||x||_{\ell^2} \cdot ||y||_{\ell^2}$$

\item Sea $\mathcal{C}[0,1]$
$$\int_{0}^{1} |f(x) \cdot g(x)| \leq \left( \int_{0}^{1} |f(x)|^2 dx \right)^\frac{1}{2} \cdot \left( \int_{0}^{1} |g(x)|^2 dx \right)^\frac{1}{2} $$
\end{itemize}

\underline{Observación}

Se dice que dos normas son equivalentes en un espacio vectorial $E$, $|| \cdot ||_1 \approx || \cdot ||_1$, si 
$$\exists c_1,c_2 > 0 : c_1 \leq ||x||_2 \leq ||x||_1 \leq c_2 ||x||_2 : \forall x \in E$$

\underline{Observación}

Dado $(E,d)$, ¿cuándo existe una norma que genera $d$, tal que $d(x,y) = ||x-y||$?

\begin{itemize}
\item $d(x+z,y+z) = d(x,y)$ (Invariante por traslaciones)

\item $d(\lambda x, \lambda y) = |\lambda | d(x,y)$ (Invariante por dilataciones)
\end{itemize}

\underline{Observación: Ley del Paralelogramo}

Sea $(E, \langle \cdot , \cdot \rangle)$, $||x||^2 = \langle x,x\rangle$, se cumple que:
$$2||x||^2 + 2||y||^2 = ||x+y||^2 + ||x-y||^2$$

Si se diese que $x \perp y \Rightarrow ||x+y|| = ||x-y||$, es decir, obtenemos el Teorema de Pitágoras:
$$||x||^2 + ||y||^2 = ||x+y||^2$$

Es interesante observar que el recíproco también es cierto. Dado un espacio normado $(E, || \cdot ||)$, si $|| \cdot ||$ satisface la Ley del Paralelogramo, entonces existe un producto escalar $\langle \cdot , \cdot \rangle$ tal que $||x||^2 = \langle x,x\rangle$. La demostración se deja como ejercicio.

\section*{TOPOLOGÍA EN ESPACIOS MÉTRICOS}
\begin{defi}[Bola]
Sea el espacio métrico $(E,d)$. Se define la \textbf{bola abierta} de centro $x\in E$ y radio $r>0$ al conjunto:
$$B(x,r) = \{y \in E : d(x,y) < r\}$$
Asimismo, definimos como \textbf{bola cerrada} al conjunto:
$$\bar{B}(x,r) = \{y \in E : d(x,y) \leq r\}$$
\end{defi}

\underline{Ejemplos}:
\begin{itemize}
\item En $\mathbb{R}$, tomando $x = \frac{a+b}{2}$ donde $a < b$ y $a,b \in \mathbb{R}$, elegimos $r = \frac{b-a}{2} > 0$. Luego ocurre que:
$$B(x,r) = \{y \in \mathbb{R} : \left| y - \frac{a+b}{2}\right| < \frac{b-a}{2}\} = (a,b)$$

\item Sea $(X,d)$, siendo $d$ la métrica discreta $d(x,y) = \begin{cases} 0 & x=y \\ 1 & x \neq y \end{cases}$, tenemos entonces que:
\begin{align*}
B(x,2) = X && B(x,1) = \{x\} && \bar{B}(x,1) = X
\end{align*}
\end{itemize}

\begin{defi}[Conjunto abierto]
Sea $(X,d)$ un espacio métrico, se dice que un conjunto $A \subset X$ es \textbf{abierto} si:
$$\forall x \in A, \exists \varepsilon > 0 : B(x,\varepsilon) \subset A$$
\end{defi}

\underline{Observaciones}:
\begin{itemize}
\item Si $\varepsilon_1 < \varepsilon_2 \Rightarrow B(x,\varepsilon_1) \subset B(x,\varepsilon_2)$

\item El conjunto vacío $\emptyset$ y $X$ son abiertos.

\item El intervalo $(0,1]$ no es abierto en $\mathbb{R}$, ya que para $x=1, \nexists \varepsilon > 0 : B(1,\varepsilon) \subset (0,1]$

\item Consideramos el espacio métrico $\left( (0,1], d_2 \right)$. El conjunto $(0,1]$ es abierto es este espacio métrico determinado. Es decir, que el hecho de ser abierto o no depende del espacio métrico en el que nos encontramos.

\item $\{x\}$ no es abierto en $\mathbb{R}$. Sin embargo, $\{x\}$ sí es abierto en $(\mathbb{R}, d_{\mbox{\tiny discreta}})$

\end{itemize}


\begin{prop}
En un espacio métrico $(E,d)$, toda bola abierta es un abierto.
\end{prop}

\begin{demo}
Sean $x \in E$ y $r>0$ tomamos $y \in B(x,r)$. Queremos probar que $\exists \varepsilon > 0 : B(y,e) \subset B(x,r)$, luego sea $\varepsilon = r - d(x,y) > 0$ vamos a ver que $B(y,e) \subset B(x,r)$. Para ello sea $z \in B(y, \varepsilon)$ comprobemos que $d(x,z) < r$:
$$d(x,z) \leq d(x,y) + d(y,z) < d(x,y) + \varepsilon = r$$
\end{demo}

\begin{prop}
Sea el espacio métrico $(E,d)$, entonces\footnote{Para la 2ª afirmación no es necesario que la familia sea finita} ocurre:
\end{prop}
\begin{enumerate}
\item $\forall A_1, \ldots, A_n$ familia de conjuntos abiertos en $E$, $\displaystyle\bigcap_{j=1}^N A_j$ es abierto en $E$.

\item $\forall \{A_\alpha\}_{\alpha \in I}$ familia arbitraria de conjuntos abiertos en $E$, $\displaystyle \bigcup_{\alpha \in I} A_\alpha$ es abierto en $E$.
\end{enumerate}

\begin{demo}
\begin{itemize}
\item Tomamos $x \in \displaystyle\bigcap_{j=1}^N A_j$ para intentar probar que:
$$\exists \varepsilon > 0 : B(x,\varepsilon) \subset \bigcap_{j=1}^N A_j$$
Para ello, la idea es que fijado un $j$ tendremos un $\varepsilon_j$ que valdrá, luego como es un conjunto finito podemos quedarnos con el más pequeño y valdrá para todos los demás:
$$j \in \{1, \ldots , N\} \Rightarrow \exists \varepsilon_j > 0 : B(x,\varepsilon_j) \subset A_j \Rightarrow \varepsilon = \underset{j \in \{1, \ldots , N\}}{\min}\{\varepsilon_j\} > 0$$
De esta forma ocurre que:
$$B(x,\varepsilon) \subset \bigcap_{j=1}^N B(x,\varepsilon_j)\subset  \bigcap_{j=1}^N A_j$$
\item Sea $x \in \displaystyle\bigcup_{\alpha \in I} A_\alpha \Rightarrow \exists \beta \in I : x \in A_\beta $ , luego se tiene que:
$$\exists \varepsilon > 0 : B(x, \varepsilon) \subset A_\beta \subset \bigcup_{\alpha \in I} A_\alpha$$
\end{itemize}
\end{demo}

\underline{Observación}:
En $(\mathbb{R}^n, d_2), A_j=(-\frac{1}{j}, \frac{1}{j})$ es abierto pero $\displaystyle\bigcap_{j=1}^\infty A_j = \{0\}$  \textbf{no} es abierto en $\mathbb{R}$.

\begin{defi}[Topología]
Toda familia de subconjuntos $T$ de un conjunto $X$, $T \subset \wp (X)$, que satisface:
\begin{itemize}
\item $\emptyset, X \in T$

\item $T$ es invariante por intersecciones finitas.

\item $T$ es invariante por uniones arbitrarias.
\end{itemize}

Se denomina como \textbf{topología de X}.
\end{defi}

 Así, la familia de abiertos de un espacio métrico $(X,d)$ es una topología.

\begin{defi}[Punto interior]
Dado un espacio métrico $(E,d)$, se dice que $x$ es un \textbf{punto interior} de $A \subset E$ si cumple que:
$$\exists \varepsilon > 0 : B(x, \varepsilon) \subset A$$
Denotamos como $\mathring{A} = int(A) = \{x \in A : x \mbox{ es punto interior}\}$
\end{defi}

\underline{Observaciones}:
\begin{itemize}
\item $\mathring{A} \subset A$
\item $A$ es abierto $\Leftrightarrow A = \mathring{A}$
\item $x \in \mathring{A} \Leftrightarrow \exists U$ abierto $: x \in U \subset A$
\item $A=\{0\}\Rightarrow \mathring{A} = \emptyset$
\item $A = [0,1]\Rightarrow\mathring{A} = (0,1)$
\item $\mathring{A} = \bigcup_{\mathcal{U} \subset A} \mathcal{U}$ donde $U$ es abierto.
\end{itemize}

\end{document}
